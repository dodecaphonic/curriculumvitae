\documentclass[11pt,a4paper]{moderncv}
\moderncvtheme[green]{casual}
\usepackage[utf8]{inputenc}
\usepackage[brazil]{babel}
\firstname{Vitor Peres}
\familyname{Capela Pereira}


\address{Rua Leitão da Cunha 48 apt. 302 -- Laranjeiras}{Rio de Janeiro}
\mobile{21 8231 0337}
\phone{21 2225 5730}
\email{dodecaphonic@gmail.com}

\quote{Interessado primordialmente em metodologias ágeis, tecnologias modernas e desenvolvimento para a web.}

\begin{document}

\hyphenation{PostgreSQL}
\hyphenation{Javascript}
\hyphenation{Passenger}
\hyphenation{balancing}

\maketitle

\section{Formação}
\cventry{2007--atualmente}{Gestão em Tecnologia da Informação}{Universidade do Sul de Santa Catarina -- UNISUL}{}{}{}

\section{Experiência}

\subsection{Vocacional}
\cventry{12/2009--atualmente}{Gerente de Desenvolvimento}{PRODEC
  Consultoria para Decisão Ltda.}{Rio de Janeiro--RJ}{}{Aplicação da
  metodologia SCRUM para gerenciamento dos projetos RodPro, Sigvia e   Conserva; consultoria em tecnologia para projeto RodPro (C\# 3.5,   WPF, Design Patterns); projeto e implementação de nova versão do Conserva (Ruby on Rails, PostgreSQL, Javascript); consultoria para projeto Sigvia (C++, GTK+, Design Patterns).}    \cventry{11/2005--12/2009}{Analista
  Programador}{PRODEC Consultoria para Decisão Ltda.}{Rio de
  Janeiro--RJ}{}{Projeto e implementação do Conserva, sistema
  especializado em conservação de elementos físicos de uma rodovia
  (C\#, ASP.Net, NHibernate, Javascript, Ajax.Net); projeto e
  implementação de funcionalidades no Sigvia (sistema de informações
  geográficas voltado para concessionárias de rodovias); projeto e
  implementação de biblioteca de abstração de acesso a bancos de
  dados, com suporte a joins, blobs e quatro backends (C++, Firebird,
  MySQL, PostgreSQL e Oracle); projeto e implementação de sistema para
  cadastro de sinalização rodoviária (Python, GTK+, Firebird);
  desenvolvimento de ferramentas de adaptação e migração de dados
  (Ruby, Python, Firebird, PostgreSQL, MySQL, Oracle, SQLite);
  desenvolvimento de biblioteca para geração de relatórios (C++,
  JasperServer, gSOAP).}

\cventry{07/2005--11/2005}{Programador}{Policentro TI S.A.}{Rio de
  Janeiro--RJ}{}{Manutenção e extensão de ferramenta interna de bug
  tracking/feature requesting e controle de atividades (ASP, Microsoft
  SQL Server); adição de funcionalidades ao software de helpdesk One
  or Zero (PHP, MySQL).}
\cventry{08/2003--02/2004}{Programador}{Policentro TI
  S.A.}{Brasília--DF}{}{Administração de bancos de dados (Oracle);
  manutenção, planejamento e implementação de funcionalidades no AFIN,
  sistema de contabilidade pública implantado em Manaus--AM e Campo
  Limpo--SP (Java, Struts, Hibernate); projeto e implementação de
  ferramentas para interação com clientes de uma VAN (PHP, Oracle);
  instalação de sistema de controle de versões baseado em CVS
  (Solaris, Linux).}

\cventry{03/2002--08/2003}{Administrador de Redes/Desenvolvedor
  Web}{Policentro TI S.A.}{Brasília--DF}{}{Administração de servidores
  (Linux, Solaris); deployment de infraestrutura de email (Qmail,
  vpopmail); administração de bancos de dados (MySQL, PostgreSQL,
  Oracle); gerenciamento de firewall (Aker); instalação de hardware
  especial em servidores (Linux, placas fibrechannel); port do sistema
  interno de pedidos de suporte de ASP + Microsoft SQL Server para PHP
  e MySQL.}

\pagebreak

\subsection{Outra}
\cventry{09/2004--01/2005}{Diretor de Arte}{Doisnovemeia Publicidade}{Brasília--DF}{}{Elaboração de conceitos visuais para peças publicitárias; ilustração (Adobe Illustrator, à mão livre); montagens e retoques fotográficos (Adobe Photoshop).}
	
\section{Línguas}
\cvline{Inglês}{Fluente}
\cvline{Holandês}{Intermediário}

\section{Habilidades}
\cvcomputer{Programação}{C, C++, Java, C\#, Python, Javascript, Ruby, Perl, PHP, bash}{Bancos de Dados}{CouchDB, MongoDB, PostgreSQL, SQLite, MySQL, Oracle, Microsoft SQL Server}
\cvcomputer{Sistemas Operacionais}{Mac OS X, (Open)Solaris, Linux, Windows}{Infraestrutura}{Apache, Mongrel, Phusion Passenger, iptables, load balancing, Qmail}

\end{document}
